%%%%%%%%%%%%%%%%%%%%%%%%%%%%%%%%%%%%%%%%%%%%%%%%%%%%%%%%%%%%%%%
%
% Welcome to Overleaf --- just edit your LaTeX on the left,
% and we'll compile it for you on the right. If you open the
% 'Share' menu, you can invite other users to edit at the same
% time. See www.overleaf.com/learn for more info. Enjoy!
%
%%%%%%%%%%%%%%%%%%%%%%%%%%%%%%%%%%%%%%%%%%%%%%%%%%%%%%%%%%%%%%%

% Inbuilt themes in beamer
\documentclass{beamer}

% Theme choice:
\usetheme{CambridgeUS}
\usepackage{amsmath}
\providecommand{\pr}[1]{\ensuremath{\Pr\left(#1\right)}}
\providecommand{\cdf}[2]{\ensuremath{\text{F}_{#1}\left(#2\right)}}
% Title page details: 
\title{Assignment 11} 
\author{Gautam Singh (CS21BTECH11018)}
\date{\today}

\begin{document}

% Title page frame
\begin{frame}
    \titlepage 
\end{frame}

% Outline frame
\begin{frame}{Outline}
    \tableofcontents
\end{frame}


% Lists frame
\section{Problem}
\begin{frame}{Problem Statement}

\textbf{(Papoulis/Pillai Exercise 5-52)}  A box contains $n$ white and $m$ black marbles. Let $X$ represent the number of draws needed for the r$^{th}$ white marble.
\begin{enumerate}
	\item If sampling is done with replacement, show that $X$ has a negative binomial distribution with parameters $r$ and $p = \frac{n}{m + n}$.
	\item If sampling is done without replacement, then show that for $r \leq k \leq m + n$,
		\begin{align}
			\pr{X = k} = \binom{k - 1}{r - 1}\frac{\binom{m + n - k}{n - r}}{\binom{m + n}{n}}
			\label{eq:qn}
		\end{align}
	\item For a given $k$ and $r$, show that the probability distribution in \eqref{eq:qn} tends to a negative binomial distribution as $n + m \to \infty$. Thus, for large population size, sampling with or without replacement is the same.
\end{enumerate}
\end{frame}


% Blocks frame
\section{Solution}

\begin{frame}{Solution}
	\begin{enumerate}		
		\item[1] Suppose $X = k$, $r \leq k \leq m + n$. Then, the $k^{th}$ marble drawn must be white. The other $r - 1$ white marbles can be drawn in any of the previous draws, and we must also have $k - r$ black marbles drawn. Since marbles are drawn with replacement, the probability of drawing a white marble at any instant is given by
			\begin{align}
				p = \frac{n}{m + n}
				\label{eq:p}
			\end{align}
			We can now get the PMF of $X$ (here, $q = 1 - p$)
			\begin{alertblock}{PMF of $X$, With Replacement}
        		\begin{align}
					\pr{X = k} =
					\begin{cases}
						\binom{k - 1}{r - 1}p^rq^{k - r}, & r \leq k \leq m + n \\
						0, & \textrm{otherwise}
					\end{cases}
					\label{eq:pmf}
		        \end{align}
    		\end{alertblock}
	\end{enumerate}
\end{frame}

\begin{frame}
	\begin{enumerate}
		\item[2] If sampling is done without replacement, the value of $p$ in \eqref{eq:p} keeps changing with each draw. We then write out the PMF as follows
			\begin{alertblock}{PMF of $X$, Without Replacement}
        		For $r \leq k \leq m + n$,
				\begin{align}
					\pr{X = k} &= \binom{k - 1}{r - 1}\frac{[n\ldots(n - r + 1)][m\ldots(m - k + r - 1)]}{[(m + n)\ldots(m + n - k + 1)]} \label{eq:c}\\
					&= \binom{k - 1}{r - 1}\frac{n!}{(n - r)!}\frac{m!}{(m + r - k)!}\frac{(m + n - k)!}{(m + n)!} \\
					&= \binom{k - 1}{r - 1}\frac{\binom{m + n - k}{n - r}}{\binom{m + n}{n}} \\
					\label{eq:pmf_norep}
		        \end{align}
    		\end{alertblock}
	\end{enumerate}
\end{frame}

\begin{frame}
	Therefore, we can write the PMF of $X$ completely as follows
	\begin{align}
		\pr{X = k} =
		\begin{cases}
			\binom{k - 1}{r - 1}\frac{\binom{m + n - k}{n - r}}{\binom{m + n}{n}}, & r \leq k \leq m + n \\
			0, & \textrm{otherwise}
		\end{cases}
		\label{eq:full-pmf}
	\end{align}
	\begin{enumerate}
		\item[3] Letting $m + n \to \infty$, and using \eqref{eq:p} and \eqref{eq:c},
			\begin{align}
				\pr{X = k} &\approx \binom{k - 1}{r - 1}\frac{n^rm^{k - r}}{(m + n)^k} \\
				&= \binom{k - 1}{r - 1}p^rq^{k - r} \sim NB(r, p)
				\label{eq:lim}
			\end{align}
			as desired.
	\end{enumerate}
\end{frame}
\end{document}
