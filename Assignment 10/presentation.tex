%%%%%%%%%%%%%%%%%%%%%%%%%%%%%%%%%%%%%%%%%%%%%%%%%%%%%%%%%%%%%%%
%
% Welcome to Overleaf --- just edit your LaTeX on the left,
% and we'll compile it for you on the right. If you open the
% 'Share' menu, you can invite other users to edit at the same
% time. See www.overleaf.com/learn for more info. Enjoy!
%
%%%%%%%%%%%%%%%%%%%%%%%%%%%%%%%%%%%%%%%%%%%%%%%%%%%%%%%%%%%%%%%

% Inbuilt themes in beamer
\documentclass{beamer}

% Theme choice:
\usetheme{CambridgeUS}
\usepackage{amsmath}
\providecommand{\pr}[1]{\ensuremath{\Pr\left(#1\right)}}
\providecommand{\cdf}[2]{\ensuremath{\text{F}_{#1}\left(#2\right)}}

% Title page details: 
\title{Assignment 10} 
\author{Gautam Singh (CS21BTECH11018)}
\date{\today}

\begin{document}

% Title page frame
\begin{frame}
    \titlepage 
\end{frame}

% Outline frame
\begin{frame}{Outline}
    \tableofcontents
\end{frame}


% Lists frame
\section{Problem}
\begin{frame}{Problem Statement}

\textbf{(Papoulis/Pillai Exercise 2-26)} Show that a set S with n elements has
\begin{align}
    \frac{n(n - 1)\ldots(n - k + 1)}{1.2{\ldots}k} = \frac{n!}{k!(n - k)!}
\end{align}
subsets of k elements.

\end{frame}


% Blocks frame
\section{Solution}

\begin{frame}{Solution}
    We begin by choosing any $k$ elements of a set consisting of $n$ elements. The first element can be chosen in $n$ ways, the second in $n - 1$ ways, and so on. The $k^{th}$ element can be chosen in $n - k + 1$ ways. However, the order of elements in a set does not matter. This gives a total of
    \begin{align}
        &\frac{n(n - 1)\ldots(n - k + 1)}{1.2{\ldots}k} \\
        =&\ \frac{n(n - 1){\ldots}1}{(1.2{\ldots}k)(1.2{\ldots}(n - k))} \\
        =&\ \frac{n!}{k!(n - k)!} = \binom{n}{k} 
        \label{eq:proof}
    \end{align}
    subsets containing $k$ elements. Here, $0 \leq k \leq n$. The Python code \texttt{codes/10\_1.py} verifies the identity.
\end{frame}

\end{document}
