%%%%%%%%%%%%%%%%%%%%%%%%%%%%%%%%%%%%%%%%%%%%%%%%%%%%%%%%%%%%%%%
%
% Welcome to Overleaf --- just edit your LaTeX on the left,
% and we'll compile it for you on the right. If you open the
% 'Share' menu, you can invite other users to edit at the same
% time. See www.overleaf.com/learn for more info. Enjoy!
%
%%%%%%%%%%%%%%%%%%%%%%%%%%%%%%%%%%%%%%%%%%%%%%%%%%%%%%%%%%%%%%%

% Inbuilt themes in beamer
\documentclass{beamer}

% Theme choice:
\usetheme{CambridgeUS}

\providecommand{\pr}[1]{\ensuremath{\Pr\left(#1\right)}}

% Title page details: 
\title{Assignment 8} 
\author{Gautam Singh (CS21BTECH11018)}
\date{\today}
\begin{document}

% Title page frame
\begin{frame}
    \titlepage 
\end{frame}

% Outline frame
\begin{frame}{Outline}
    \tableofcontents
\end{frame}


% Lists frame
\section{Problem}
\begin{frame}{Problem Statement}

\textbf{(NCERT Class 12, Exercise 13.2.16 )} In a hostel, 60\% of the students read Hindi newspaper, 40\% read English newspaper and 20\% read both Hindi and English newspapers. A student is selected at random.

\begin{enumerate}[label=(\alph{enumi})]
	\item Find the probability that she reads neither Hindi nor English newspapers.
	\item If she reads Hindi newspaper, find the probability that she reads English newspaper.
	\item If she reads English newspaper, find the probability that she reads Hindi newspaper.
\end{enumerate}

\end{frame}


% Blocks frame
\section{Solution}
\begin{frame}{Solution}
    \begin{block}{Events}
        \begin{enumerate}
        		\item E: Student reads English newspaper
        		\item F: Student reads Hindi newspaper
        \end{enumerate}
    \end{block}
    \begin{block}{Given}
		\begin{enumerate}
			\item $\pr{E} = 0.4$
			\item $\pr{F} = 0.6$
			\item $\pr{EF} = 0.2$
			\label{given}	
		\end{enumerate}		    
    \end{block}
    \begin{alertblock}{To find}
    		\begin{enumerate}
    			\item $\pr{E' + F'}$
    			\item $\pr{E|F}$
    			\item $\pr{F|E}$
    		\end{enumerate}
    \end{alertblock}
\end{frame} 

\begin{frame}
	\begin{block}{1. $\pr{E' + F'}$}
		Using De-Morgan's Laws,
		\begin{align}
			\pr{E' + F'} &= \pr{(EF)'} \\
			&= 1 - \pr{EF} \\
			&= 1 - 0.2 = \frac{4}{5}
			\label{sol1}		
		\end{align}
	\end{block}
	\begin{block}{2. $\pr{E|F}$}
		\begin{align}
			\pr{E|F} &= \frac{\pr{EF}}{\pr{F}} \\
			&= \frac{0.2}{0.6} = \frac{1}{3}
			\label{sol2}		
		\end{align}
	\end{block}
\end{frame}

\begin{frame}
	\begin{block}{3. $\pr{F|E}$}
		\begin{align}
			\pr{F|E} &= \frac{\pr{EF}}{\pr{E}} \\
			&= \frac{0.2}{0.4} = \frac{1}{2}
			\label{sol3}
		\end{align}
	\end{block}
	\begin{alertblock}{Answers}
    		\begin{enumerate}
    			\item $\pr{E' + F'} = \frac{4}{5}$
    			\item $\pr{E|F} = \frac{1}{3}$
    			\item $\pr{F|E} = \frac{1}{2}$
    		\end{enumerate}
    \end{alertblock}
\end{frame}

\end{document}
