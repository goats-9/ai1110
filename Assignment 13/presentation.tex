%%%%%%%%%%%%%%%%%%%%%%%%%%%%%%%%%%%%%%%%%%%%%%%%%%%%%%%%%%%%%%%
%
% Welcome to Overleaf --- just edit your LaTeX on the left,
% and we'll compile it for you on the right. If you open the
% 'Share' menu, you can invite other users to edit at the same
% time. See www.overleaf.com/learn for more info. Enjoy!
%
%%%%%%%%%%%%%%%%%%%%%%%%%%%%%%%%%%%%%%%%%%%%%%%%%%%%%%%%%%%%%%%

% Inbuilt themes in beamer
\documentclass{beamer}

% Theme choice:
\usetheme{CambridgeUS}
\usepackage{amsmath}
\usepackage{mathtools}
\newcommand\laplace{\ensuremath{\stackrel{\mathclap{\mbox{$\mathcal{L}$}}}{\longleftrightarrow}}}
\providecommand{\pr}[1]{\ensuremath{\Pr\left(#1\right)}}
\providecommand{\cdf}[2]{\ensuremath{\text{F}_{#1}\left(#2\right)}}
% Title page details: 
\title{Assignment 13} 
\author{Gautam Singh (CS21BTECH11018)}
\date{\today}

\begin{document}

% Title page frame
\begin{frame}
    \titlepage 
\end{frame}

% Outline frame
\begin{frame}{Outline}
    \tableofcontents
\end{frame}

\section{Problem}
\begin{frame}{Problem Statement}
	\textbf{(Papoulis/Pillai, Exercise 5-48)} The random variable $X$ is $N(0; \sigma^2)$.
	\begin{enumerate}
		\item Using Laplace transforms, show that if $g(x)$ is a function such that $g(x)\exp{(-\frac{x^2}{2\sigma^2})} \to 0$ as $|x| \to \infty$, then (Price's Theorem)
			\begin{align}
				\frac{dE\{g(X)\}}{dv} = \frac{1}{2}E\left\{\frac{d^2g(X)}{dX^2}\right\} 
				\label{eq:price}
			\end{align}
		\item The moments $\mu_n$ of $X$ are functions of $v$. Using \eqref{eq:price}, show that
			\begin{align}
				\mu_n(v) = \frac{n(n - 1)}{2}\int_{0}^{v}\mu_{n - 2}(\beta)d\beta
				\label{eq:moments}
			\end{align}
	\end{enumerate}
	Here, $v = \sigma^2$.
\end{frame}

\section{Properties of Normal Random Variables}
\begin{frame}{PDF and Partial Derivatives of $N(\mu; \sigma^2)$}
	Since $X \sim N(\mu; \sigma^2)$, the PDF of $X$ is given by
		\begin{align}
			f(x) = \frac{1}{\sqrt{2\pi}\sigma}\exp{\left(-\frac{(x - \mu)^2}{2\sigma^2}\right)} \\
			f(x, v) = \frac{1}{\sqrt{2\pi v}}\exp{\left(-\frac{(x - \mu)^2}{2v}\right)}
			\label{eq:gauss-pdf}
		\end{align}
	Notice that
		\begin{align}
			\frac{\partial f}{\partial v} &= \frac{\exp{\left(\frac{(x - \mu)^2}{2v}\right)}((x - \mu)^2 - v)}{2\sqrt{2\pi v^5}} \\
			\frac{\partial^2 f}{\partial x^2} &= \frac{\exp{\left(\frac{(x - \mu)^2}{2v}\right)}((x - \mu)^2 - v)}{\sqrt{2\pi v^5}} \\
			\implies \frac{\partial f}{\partial v} &= \frac{1}{2}\frac{\partial^2 f}{\partial x^2}
			\label{eq:gauss-derivatives}
		\end{align}
\end{frame}
		
\section{Laplace Transform}
\begin{frame}{Definition and Properties of Laplace Transform}
	The Laplace Transform of a real function $f(t)$ is given by
	\begin{align}
		\mathcal{L}\{f\}(s) = \int_{0}^{\infty}e^{-st}f(t)dt  
		\label{eq:lap}
	\end{align}
	where $s \in \mathbb{C}$. Here, $k(s, t) = e^{-st}$ is called the kernel function. A useful property of Laplace transforms is the following, where the $n^{\text{th}}$ derivative of $f$ is assumed to be of exponential type.
	\begin{align}
		\mathcal{L}\{f^{(n)}\} = s^{n}\mathcal{L}\{f\} - \sum_{k = 1}^{n}s^{n - k}f^{(k - 1)}(0^+)
		\label{eq:lap-diff}
	\end{align}
	For $n = 1$,
	\begin{align}
		\mathcal{L}\{f'\} = s\mathcal{L}\{f\} - f(0^+)
		\label{eq:lap-diff-one}
	\end{align}
\end{frame}

\section{Solution}
\begin{frame}{Solution}
	In this solution, we assume that $g^{(k)}(x)\exp{(-\frac{x^2}{2\sigma^2})} \to 0$ as $|x| \to \infty$ for $k = 0, 1, 2$. Applying a Laplace transform, and noting that at $f(x, 0^+) = 0$,
	\begin{align}
		\frac{dE\{g(X)\}}{dv} &\laplace s\mathcal{L}\{f\} \\
		&= s\int_{0}^{\infty}e^{-sv}\left(\int_{-\infty}^{\infty}g(x)f(x, v)dx\right)dv \\
		&= \int_{-\infty}^{\infty}g(x)\left(\int_{0}^{\infty}se^{-sv}f(x, v)dv\right)dx \\
		&= -\int_{-\infty}^{\infty}g\left(\frac{\partial k}{\partial v}fdv\right)dx \\
		&= \int_{-\infty}^{\infty}g\left(e^{-sv}\frac{\partial f}{\partial v}dv\right)dx
		\label{eq:sol1-1}
	\end{align}
\end{frame}

\begin{frame}
	Using \eqref{eq:gauss-derivatives}, and integrating repeatedly by parts using our assumptions
	\begin{align}
		\frac{dE\{g(X)\}}{dv} &\laplace \frac{1}{2}\int_{0}^{\infty}e^{-sv}\left(\int_{-\infty}^{\infty}g\frac{\partial^2 f}{\partial x^2}dx\right)dv \\
		&= \frac{1}{2}\int_{0}^{\infty}e^{-sv}\left(\int_{-\infty}^{\infty}f\frac{\partial^2 g}{\partial x^2}dx\right)dv \\
		&= \mathcal{L}\{E\left\{\frac{d^2g(X)}{dX^2}\right\}\}(s) \laplace \frac{1}{2}E\left\{\frac{d^2g(X)}{gX^2}\right\}
		\label{eq:sol1-2}
	\end{align}
\end{frame}

\begin{frame}
	For the second part, observe that $\mu_n = E[X^n]$ and hence it is a function of $v$. Further, using the exponential power series, note that for any positive integer $n$,
	\begin{align}
		\exp{\left(\frac{x^2}{2\sigma^2}\right)} &> \frac{x^{2n}}{(2\sigma^2)^nn!} \\
		\implies 0 < \frac{x^n}{\exp{\left(\frac{x^2}{2\sigma^2}\right)}} &< \frac{(2\sigma^2)^nn!}{x^n}
		\label{eq:sandwich}
	\end{align}
	and using the Sandwich Theorem, we can choose $g(x) = x^n$ to use in \eqref{eq:price}.
	\begin{align}
		\mu_n'(v) = \frac{1}{2}E\left\{n(n - 1)x^{n - 2}\right\} = \frac{n(n - 1)}{2}\mu_{n - 2}{(v)}
		\label{eq:de}
	\end{align}
	However, note that if $v = 0$, then from \eqref{eq:gauss-pdf}, $f(x) = 0$ and consequently $\mu_n(0) = 0$. Integrating \eqref{eq:de} and changing variables, we get
	\begin{align}
		\mu_n(v) = \frac{n(n - 1)}{2}\int_{0}^{v}\mu_{n - 2}(\beta)d\beta
		\label{eq:sol2}
	\end{align}
\end{frame}

\end{document}
