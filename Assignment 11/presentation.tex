%%%%%%%%%%%%%%%%%%%%%%%%%%%%%%%%%%%%%%%%%%%%%%%%%%%%%%%%%%%%%%%
%
% Welcome to Overleaf --- just edit your LaTeX on the left,
% and we'll compile it for you on the right. If you open the
% 'Share' menu, you can invite other users to edit at the same
% time. See www.overleaf.com/learn for more info. Enjoy!
%
%%%%%%%%%%%%%%%%%%%%%%%%%%%%%%%%%%%%%%%%%%%%%%%%%%%%%%%%%%%%%%%

% Inbuilt themes in beamer
\documentclass{beamer}

% Theme choice:
\usetheme{CambridgeUS}
\usepackage{amsmath}
\providecommand{\pr}[1]{\ensuremath{\Pr\left(#1\right)}}
\providecommand{\cdf}[2]{\ensuremath{\text{F}_{#1}\left(#2\right)}}

% Title page details: 
\title{Assignment 11} 
\author{Gautam Singh (CS21BTECH11018)}
\date{\today}
\logo{\large \LaTeX{}}


\begin{document}

% Title page frame
\begin{frame}
    \titlepage 
\end{frame}

% Remove logo from the next slides
\logo{}


% Outline frame
\begin{frame}{Outline}
    \tableofcontents
\end{frame}


% Lists frame
\section{Problem}
\begin{frame}{Problem Statement}

\textbf{(Papoulis/Pillai Exercise 3-4)} A coin with $\pr{h} = p = 1 - q$ is tossed n times. Show that the probability that the number
of heads is even equals $0.5[1 + (q - p)^n]$. 

\end{frame}


% Blocks frame
\section{Solution}

\begin{frame}{Solution}
    Let $Y \sim$ Bin($n$, $p$) represent the binomial random variable representing the number of heads attained. We note the following binomial expansions:
    \begin{align}
        1 = (q + p)^n &= \sum_{k = 0}^{k = n}\binom{n}{k}q^{(n - k)}p^k \\
        &= \binom{n}{0}q^n + \binom{n}{1}q^{(n - 1)}p + \binom{n}{2}q^{(n - 2)}p^2 + \ldots \label{eq:t1}\\
        (q - p)^n &= \sum_{k = 0}^{k = n}\binom{n}{k}(-1)^kq^{(n - k)}p^k \\
        &= \binom{n}{0}q^n - \binom{n}{1}q^{(n - 1)}p + \binom{n}{2}q^{(n - 2)}p^2 + \ldots \label{eq:t2}
    \end{align}
\end{frame}

\begin{frame}
    \begin{alertblock}{PMF of Y}
        \begin{align}
            \pr{Y = k} &= 
            \begin{cases}
                \binom{n}{k}p^{(n - k)}q^k, & 0 \leq k \leq n \\
                0, & \textrm{otherwise} 
            \end{cases}  \\
            \implies \pr{Y \equiv 0\ (\textrm{mod } 2)} &= \binom{n}{0}p^n + \binom{n}{2}p^{(n - 2)}q^2 + \ldots
        \end{align}
    \end{alertblock}
    Therefore, adding \autoref{eq:t1} and \autoref{eq:t2}, we get,
    \begin{align}
        1 + (q - p)^n &= 2(\binom{n}{0}p^n + \binom{n}{2}p^{(n - 2)}q^2 + \ldots) \\ 
        &= 2\pr{Y \equiv 0\ (\textrm{mod } 2)} \\
        \implies \pr{Y \equiv 0\ (\textrm{mod } 2)} &= 0.5[1 + (q - p)^n]
        \label{eq:proof}
    \end{align}
    as deisred. This is verified in \texttt{codes/11\_1.py}.
\end{frame}
\end{document}
